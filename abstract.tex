% $Log: abstract.tex,v $
% Revision 1.1  93/05/14  14:56:25  starflt
% Initial revision
%
% Revision 1.1  90/05/04  10:41:01  lwvanels
% Initial revision
%
%
%% The text of your abstract and nothing else (other than comments) goes here.
%% It will be single-spaced and the rest of the text that is supposed to go on
%% the abstract page will be generated by the abstractpage environment.  This
%% file should be \input (not \include 'd) from cover.tex.

Stellar ages will play a big role in the next generation of astronomy.
Useful to exoplaneteers and galactic archaeologists alike, this relatively
under-exploited stellar property is currently limited by the precision of
dating techniques.
The work presented in this thesis contributes incrementally to a greater
understanding of rotation period decay in Main Sequence (MS) stars as a proxy
for stellar age.
Inferring stellar ages from rotation periods, `gyrochronology', is the {\it
only} dating method with the potential to provide ages for stars on the
hundreds-of-thousands scale.
Unfortunately however, it suffers from being poorly calibrated as the sample
of cool, MS stars with precise ages is extremely sparse.
Using light curves of spotted, rotating, MS, FGK stars with asteroseismic ages
from the \kepler\ spacecraft, I attempted to recalibrate the relation between
rotation period, colour and age.
I demonstrate that the simple, `straight line' gyrochronology relations used
in the past are unable to explain the new asteroseismic sample.
Questions are raised about the power of gyrochronology---does it accurately
predict ages for old stars?
To answer fully this question, it will be necessary to exploit new data from
the \ktwo\ (the repurposed \kepler\ mission).
\ktwo\ has observed (and is still observing) several open clusters and
asteroseismic field stars which may provide new insight into stellar
rotational evolution.
Unfortunately, systematic features in \ktwo\ light curves produced by
\kepler's reduced pointing precision inhibit the detection of astrophysical
signals in the data.
These systematic features can be removed by modelling and subtracting them
from the time series, `detrending', but this process can remove some signals
and can even inject noise.
For this reason I developed a method for detecting periodic signals in \ktwo\
light curves without detrending: the Systematics-Insensitive Periodogram
(SIP).
This method is particularly useful for red giant asteroseismology.
Precise ages can be inferred for oscillating red giants using the SIP and will
be useful for galactic archaeology and open cluster age inference.
In the next chapter of this thesis I return to the problem of stellar rotation
period inference.
Current methods for rotation period inference can produce inaccurate,
imprecise periods with poorly approximated uncertainties and often without
uncertainties altogether.
I present a new method for inferring precise, accurate, probabilistic rotation
periods with accurate uncertainties using Gaussian processes.
Although expensive to compute, this method is ideal for applying to individual
targets.
I hope to continue to develop this method and apply it to a large ensemble
light curves from \kepler\ and other photometric surveys in the future.
Star spots and acoustic (p-mode) oscillations are not the only mechanisms
that produce variability in dwarfs and giants.
A combination of asteroseismic pulsations and granulation on the stellar
surface produces variability on short timescales.
It has been shown that the amplitude of this short-term variability, called
`flicker' is strongly correlated with both surface gravity and stellar
density \citep{Bastien2013, Bastien2016, Kipping2014}.
However, there is substantial additional scatter in these relations that is
not accounted for by the observational uncertainties.
I provide a new calibration of these relations which models this level of
additional, astrophysical variance using hierarchical probabilistic inference.
In the final chapter of this thesis I explore rotation period recovery with
the Large Synoptic Survey Telescope (\LSST).
With its ten year baseline, \LSST\ light curves will be sensitive to long
rotation periods which are characteristic of old and low-mass stars.
If the rotation periods of such stars can be measured from \LSST\ light
curves, it may be possible to improve the gyrochronology relations.
We find that \LSST\ is most sensitive to rotation periods between 10 and 20
days.
Its sensitivity falls at short periods due to the sparsity of its sampling and
at longer periods due to the lower amplitudes of variability and smaller
apparent magnitudes of slow rotators.
