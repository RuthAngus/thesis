\Chapter{Conclusions}
\label{chapter:conclusions}

In chapter \ref{chapter:gyro} I used new results from \kepler\
asteroseismology to recalibrate the relation between rotation period, age and
colour.
The short cadence \kepler\ light curves of several cool dwarf, Solar-like
oscillators were analysed by \citet{Chaplin2014} who provide a catalogue of
ages for these targets.
We used photometric rotation periods of these stars published in
\citet{Garcia2014}.
They used ACF and wavelet methods to measure the periodicity of star
spot-induced brightness modulations in their \kepler\ light curves.
We fit a gyrochronology model to these data, of the form $P = A^n \times
a(B-V-c)^b$, where $P$ is rotation period (in days), $A$ is age (in Myr), $B$
and $V$ are $B$ and $V$ band magnitudes respectively and $a$, $b$, $c$ and $n$ are
dimensionless free parameters.
We found evidence to suggest that this function does not provide a good fit to
the data and that some old \kepler\ asteroseismic targets are more rapidly
rotating that expected, given their age and colour.
\citet{Vansaders2016} present an adaptation to their theoretical model which
{\it is} able to reproduce the trends in this data set.
They suggest that there is a critical Rossby number ($Ro=2.16$) at which the
magnetic dynamo that drives angular momentum loss shuts off.
New rotation periods from the repurposed \kepler\ mission, \ktwo\ may shed
light on this controversial topic.

In chapter \ref{chapter:sip} I present a method for searching for periodic
signals (\eg rotation periods) in \ktwo\ light curves without detrending.
The pointing precision of \kepler\ was dramatically reduced when its third
reaction wheel broke.
\ktwo\ light curves are contaminated with high amplitude systematic features
as a result.
Accurately modelling the noise in these light curves is essential in order to
search for exoplanets, perform asteroseismology or measure stellar rotation
periods.
We constructed a noise model by decomposing all \ktwo\ light curves from
campaign 1 into a set of orthogonal basis vectors called `Eigen Light Curves'
(ELCs).
Instead of detrending, fitting the noise model and subtracting it from the
data, we model the noise and the signal simultaneously and marginalise over
the noise model.
The Systematics-Insensitive Periodogram (SIP) is a method that uses a linear
combination of 150 of these ELCs to model a light curve while simulataneously
fitting a sinusoid to the data at a given frequency.
Finding the amplitudes of the best-fit sinusoids over a grid of frequencies
produces a SIP.
The SIP is particularly effective for red giant asteroseismology as these
signals are typically sinusoidal and plagued by a six-hour thruster firing
signal that most detrending algorithms are unable to remove.
The capabilities of the SIP are limited for rotation period inference as
it is difficult to separate systematics from physical signals on long
timescales and because a sinusoid is an imperfect model for stellar rotation.

In chapter \ref{chapter:GP} I present a new method for inferring precise and
accurate rotation periods from \kepler\ light curves using Gaussian processes.
A Gaussian process model is a flexible, semi-parametric model that is capable
of capturing the non-sinusoidal, quasi-periodic nature of stellar light
curves.
I compare the GP method to the Lomb-Scargle periodogram and ACF methods,
finding it to be more accurate and precise than both.
In addition it provides accurate uncertainties because, in our implementation,
a full posterior PDF of the rotation period parameter is explored.
Unfortunately, while this method works well on noise-free simulations, it is
currently extremely computationally expensive when applied to noisy light
curves.
It is therefore currently most useful for individual targets rather than large
ensembles of light curves.

In chapter \ref{chapter:flicker} of this thesis I use \kepler\ flicker
measurements from \citet{Bastien2013} and fundamental stellar parameters from
asteroseismology to recalibrate the relation between flicker, surface gravity
and stellar density.
These relations are stochastic, although they have been treated
deterministically in the past \citep[\eg][]{Bastien2013, Kipping2014} and
there is significant intrinsic scatter that is not accounted for by the
observational uncertainties.
I use hierarchical bayesian modelling to quantify  model these relations and
quantify the additional scatter required to explain the observations.

Finally, in chapter \ref{chapter:future} I investigate opportunities for
inferring rotation periods from \LSST\ light curves.
\LSST\ data will be sparse and unevenly sampled---quite the opposite of
\kepler.
If we can detect long rotation periods of old, low-mass stars in \LSST's ten
years of observations, currently empty parts of the rotation period-mass-age
parameter space will be filled in, with promising implications for
gyrochronology.

I began this thesis by introducing the \kepler\ spacecraft which has provided
all the data used in this work.
I would like to conclude by saying that, in my opinion, \kepler's legacy has
to be the greatest of any single purpose astronomical instrument ever built;
not just for exoplanets but for stellar astrophysics too.
\kepler\ has revived the astronomical community's general interest in stars:
after all, one can only understand an exoplanet as well as one understands the
star it orbits.
Were it not for \kepler, this thesis would look very different and I, for one,
am very grateful for the unique challenges posed by its rich, unrivalled data
set.
