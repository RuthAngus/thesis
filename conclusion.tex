\chapter{Conclusions}

information, so these will be used too) and will be tested on binary stars.
In addition to the basic input catalogues available for the transit survey
stars, I will gather information from every other available source.
Light curves themselves are rich in age information, for example, even where
a rotation period cannot be measured, the {\it absence} of discernable
variability is indicative of old-age.
I will use the method of \citet{bastien} to measure short-timescale brightness
fluctuations (known as flicker) which are correlated with surface gravity.
I will also use new distance and proper motion information from GAIA as soon as
the data become available (expected in early 2017).
To obtain accurate gyrochronological ages, reliable rotation periods must be
inferred---this is another challenge which has prohibited the exploration of
time-dependent exoplanet populations thus far.
Standard methods for rotation period inference such as autocorrelation
functions (ACFs) and sine-fitting periodograms are not optimal for noisy light
curves with low-amplitude signals, are not probabilistic, do not provide
realistic uncertainties and are performed on `detrended' \Kepler\ light curves
with significantly reduced power at periods of 30 days and above.
My new technique for measuring probabilistic rotation periods using Gaussian
processes (GPs) can extract rotation periods from low-amplitude signals that
are buried in noisy data, can be applied directly to unprocessed light curves
and measures more precise and accuration rotation periods than ACF and
periodogram methods \citep{AngusIAU}, see figure \ref{fig:rotation}.
I will use this method to compute a probabilistic constraint on the rotation
periods of {\it all} transit survey stars.

This new dating method will be probabilistic---hierarchical Bayesian inference
will be performed in order to `learn' an informative age prior from the data
themselves; a prior that will reflect the most probable age of a star, given
the age distribution of other stars observed.

There are several advantages to this dating model.
Firstly, combining all the information will provide more precise and accurate
ages than any one method used on its own, and, by definition, these ages will
be less dependent upon one single model.
Secondly, it will provide an age constraint for every star observed by
a photometric survey since {\it some} age information is always available.
Thirdly, it will be directly compatible with my probabilistic, GP rotation
period measurement method and finally, it will be easily updated as new data
become available.
By modelling stellar ages using all available information and combining
multiple dating techniques, my hierarchical, Bayesian dating method with
informative priors {\bf will provide the most precise and accurate ages ever
inferred for every star observed by \Kepler, \Ktwo\ and \TESS}.

This project is highly ambitious and will have enormous impact if successful,
however even if there is no detectable age-trend in the \Kepler\ data it will
lead to other discoveries and useful products for the astronomical
community.
Firstly, a rotation period and age for every \Kepler, \Ktwo\ and \TESS\ star
will be beneficial to both exoplaneteers and galactic archaeologists alike,
\citep[e.g.][]{bovy}.
Secondly, stellar rotation periods can reveal potential star-planet
interactions in which a planet transfers angular momentum to its host---I
intend to unambiguously confirm this phenomenon.
Finally, this work will pave the way for exoplanet-age studies with PLATO.
Launching in 2025, this space telescope will provide highly precise
asteroseismic ages for thousands of stars, revolutionising the field of
stellar ages.

