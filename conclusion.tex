\chapter{Conclusions}
\label{chapter:conclusions}

The work in this thesis contributes to our understanding of stars.

Since the work in \ref{chapter:gyro} was published as \citet{Angus2015}, the
result seen here: that \kepler\ asteroseismic stars rotate unexpectedly
rapidly given their age and mass, was also found by \citet{Vansaders2016}.
Instead of using the entire ensemble of short cadence asteroseismic targets
from \citet{Chaplin2014}, they used only `boutique' targets.
These are the highest S/N stars with individual oscillation mode frequencies
detectable in their power spectra.
These targets are not well described by existing gyrochronology
models---whereas I showed that the \citet{Barnes2007} empirical form did not
fit the data, \citet{Vansaders2016} compare rotation periods to predictions
made using their theoretically motivated model, equation \ref{eq:vansaders} in
chapter \ref{chapter:intro}.
They also cannot reproduce the observations using their model.
They present the following alteration to their model which {\it does} describe
the data:
\begin{equation}
\frac{dJ}{dt} = \left\{
                \begin{array}{ll}
                  f_K K_M \omega \left( \frac{\omega_{crit}}{\omega_\odot}
                  \right)^2, \omega_{crit} \leq \omega
                  \frac{\tau_{c}}{\tau_{c, \odot}}, Ro \leq Ro_{crit} \\
                  f_K K_M \omega \left( \frac{\omega\tau_{c}}
                  {\omega_\odot\tau_{c, \odot}}
                  \right)^2, \omega_{crit} > \omega
                  \frac{\tau_{c}}{\tau_{c, \odot}}, Ro \leq Ro_{crit} \\
                  0, Ro > Ro_{crit}
                \end{array}
              \right.,
\end{equation}
where $K_M$ is defined in equation \ref{eq:vansaders2}.
By introducing a threshold Rossby number, $Ro_{crit}$, above which there is
no angular braking, they are able to reproduce the observations.
By fitting their model to the \kepler\ targets and the Sun they find
$Ro_{crit} = Ro_\odot = 2.16$.

In order to test this theory it is imperative that we obtain new
observations---in particular, rotation periods of old stars with precise and
reliable ages.
